\documentclass[12pt,a4paper]{article}
% FIXME: UTF8
\usepackage[latin1]{inputenc}    % Ascii-Format dieses Dokuments
\usepackage{german}
\usepackage{longtable}           % lange Tabellen
%\usepackage[dvips]{epsfig}

\begin{document}
\title{gui2user}
\date{2008-08-17 v0.3-triemli}
\author{apic, ilo, telmich}

\maketitle

\section{Introduction}
This specification lists Commands to be supported by all GUIs. It
specifies, how the GUI presents itself to the user (gui2user).

\subsection{Writing convention}
Parameters enclosed in $<$ and $>$ must be specified to the command.
Parameters enclosed in $[$ and $]$ can be specified optionally .

\subsection{Minimal philosophy}
This specifications describes all the commands that \textbf{must} be
supported by every client. This means clients may have additional possibilities
to send commands via different methdos (like clicking on a button or via
voice input), but they must support all of the commands listed, because this
way the user always has the familar commands available.

\subsection{Changelog}
\subsubsection{v0.1-mrmcd110b to v0.2-train}
   \begin{itemize}
      \item Renamed \textbf{mslist} to \textbf{mlist}, so it is consequently the prefix
      "`m"' for marktschreier.
      \item Some minor language cleanup 
      \item Added "`minimal philosophy"'
      \item Added minimal writing convention
      \item Added /peer commands
      \item Gave the document some structure
      \item \textbf{Change by}: telmich
   \end{itemize}

\section{Commands}
All commands begin with a "`/"' as first character.

%---------------------------------------------------------------------
\subsection{Command length}
The shall only accept commands up to a langth of 256 Byte.
If the user inputs longer commands, they must be truncated
to 256 Bytes.
%---------------------------------------------------------------------
\subsection{Version 0.1 commands}
Those commands may need to be reworked.
%---------------------------------------------------------------------
\subsubsection{Send text}
If there is no /, it should be assumed that it is a message.
The text will be send to the currently seleceted channel or nick.

%---------------------------------------------------------------------
\subsubsection{/join $<marktschreier>$ $<channel>$}
Join the specific channel on the specified marktschreier.

%---------------------------------------------------------------------
\subsubsection{/list $<markschreier>$}
List available channels on specified marktschreier.

%---------------------------------------------------------------------
\subsection{/mlist}
List connected marktschreier.

%---------------------------------------------------------------------
\subsection{/mconnect $<marktschreier>$}
Connect to marktschreier.

%---------------------------------------------------------------------
\subsection{/whois $<nick>$}
Display detailled information about a nick:
\begin{itemize}
\item PGP fingerprint
\item Full name
\item E-Mail
\end{itemize}

%---------------------------------------------------------------------
\subsubsection{/clist $<channel>$}
Display nicks in the specified channel.

%---------------------------------------------------------------------
\subsubsection{/names}
Display nicks in the currently active channel.

%---------------------------------------------------------------------
\subsubsection{/plist}
List currently known peers.

%---------------------------------------------------------------------
\subsubsection{/leave $<channel>$}
Leave a channel.

%---------------------------------------------------------------------
\subsection{General commands}
\subsubsection{/quit}
Tell ceof to quit, ceof tells all GUIs and subcomponents to quit and quits.

%---------------------------------------------------------------------
\subsection{Peer commands}
Peer related commands.

%---------------------------------------------------------------------
\subsubsection{/peer add $<$name$>$ $<$initial address$>$}
Add the peer as \textit{name} to the list of known peers
and connect to it, to retrieve other addresses.
\begin{itemize}
\item \textit{name} is 128 Bytes, including 0-byte termination.
If the name is shorter it is padded with 0-bytes.
\item \textit{initial address} is 128 Bytes, including 0-byte termination.
If the initial address is shorter it is padded with 0-bytes.
\end{itemize}
%---------------------------------------------------------------------
\subsubsection{/peer rename $<$oldname$>$ $<$newname$>$}
Renames the peer.
%---------------------------------------------------------------------

\subsection{Alias}
Those commands are provided for easier integration of
new people using EOF.
%---------------------------------------------------------------------
\subsection{/msg $<nick>$ $<message>$}
Is an alias for \textit{/peer send $<$name$>$ $<$message$>$}
%---------------------------------------------------------------------

\end{document}
