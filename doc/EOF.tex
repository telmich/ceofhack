\documentclass[12pt,a4paper]{article}
\usepackage[latin1]{inputenc}    % Ascii-Format dieses Dokuments
\usepackage{longtable}           % lange Tabellen

\begin{document}
\title{EOF\\Eris Onion Forwarding\\
The secure, peer-to-peer, decentralised anonymous chat network}
\date{2009-03-12}
\author{Nico -telmich- Schottelius}

\maketitle
\tableofcontents
\newpage

\section{Abstract}
EOF is chat protocol that supports secure chatting.
Secure chatting consists of the following features:
\begin{enumerate}
\item Nobody, but the intended receiver(s) know \emph{that} you wrote a message.
\item Nobody, but the intended receiver(s) can view the \emph{message content}.
\item Nobody, but the intended receiver(s) can \emph{verify} the source of the message being you.
\item Nobody, but the intended receiver(s) know \emph{who} you sent a message to.
\end{enumerate}
We don't think it's possible to hide that you are part of the chat network.
Thus, every participant of an EOF network will constantly send packets
(so called \emph{noise}) in a defined frequency (for instance every 250 ms).
The noise is also used to defend against timing analysis.
In case you are sending out a message, the message packet will be added to the
queue and sent with the next free interval.

We encrypt and sign every message via public-key cryptography\cite{pgp-1},
so that only the receiver can decrypt, view the message content and verify
the message sender.

The message packets are always send indirectly via onion routing\cite{onion-1}.
The idea is taken from the Tor project\cite{tor-1}, though EOF uses an enhanced
version: EOF does not know about entry or exit nodes. If you are the intended
receiver you may or may not continue to forward the message, which is defined
by the sender of the message. That said, EOF must use source
routing\cite{source-routing-1}.

EOF knows about two different chat destinations:
\begin{enumerate}
\item Peers
\item Groups of peers
\end{enumerate}
A peer is just another person (direct chat), a group of peers is the EOF
equivalent of the IRC channel\cite{irc-1}. As there is no central server,
groups of peers are managed by each client and may thus by differnt by
different peers.

EOF does not implement nor specify \emph{transport protocols} itself.
The EOF community is urged to implement them in a creative way: Usage
of well-known protocols like TCP\cite{tcp-1}, HTTP\cite{http-1},
SMTP\cite{smtp-1} or even transmission of packets on avian
carriers\cite{avian-1} are encouraged. The tunneling of EOF packets into
those protocols (also know as obfuscation) makes it harder to detect
and block EOF traffic.
% -----------------------------------------------------------------------------
\appendix
\section{Sources}
\begin{thebibliography}{42}
\bibitem{pgp-1} http://en.wikipedia.org/wiki/Public-key\_cryptography
\bibitem{onion-1} http://en.wikipedia.org/wiki/Onion\_routing
\bibitem{tor-1} https://wiki.torproject.org/noreply/TheOnionRouter
\bibitem{irc-1} RFC 1459: http://www.irchelp.org/irchelp/rfc/rfc.html
\bibitem{source-routing-1} http://en.wikipedia.org/wiki/Source\_routing
\bibitem{tcp-1} RFC 793: http://www.faqs.org/rfcs/rfc793.html
\bibitem{http-1} RFC 2616: http://www.faqs.org/rfcs/rfc2616.html
\bibitem{smtp-1} RFC 2821: http://www.faqs.org/rfcs/rfc2821.html
\bibitem{avian-1} RFC 1149: http://www.faqs.org/rfcs/rfc1149.html
\end{thebibliography}
\end{document}
