\documentclass[12pt,a4paper]{article}
\usepackage[latin1]{inputenc}    % Ascii-Format dieses Dokuments
\usepackage{longtable}           % lange Tabellen

\begin{document}
\title{EOF\\Eris Onion Forwarding\\The secure, p2p, anonymous chat network}
\date{2009-03-12}
\author{Nico -telmich- Schottelius}

\maketitle
\tableofcontents
\newpage

\section{Abstract}
EOF is chat protocol that supports secure chatting.
Secure chatting consists of the following features:
\begin{itemize}
\item Nobody, but the intended receiver(s) know that you wrote a message.
\item Nobody, but the intended receiver(s) can view the message content.
\end{itemize}
We don't think it's possible to hide that you are part of the chat network.
Thus, every participant of an EOF network will constantly send packets
(so called \emph{noise}) in a defined frequency (for instance every 250 ms).
In case you are sending out a message, the message packet will be added to the
queue and sent with the next free interval.

% -----------------------------------------------------------------------------
\end{document}
