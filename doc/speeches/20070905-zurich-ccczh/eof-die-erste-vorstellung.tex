% first presentation about cmtp
\documentclass{beamer}
\usepackage{german}
\usepackage{beamerthemesplit}
\setbeamercovered{dynamic}
\usetheme{Malmoe}

\title{EOF}
\subtitle{Die erste Vorstellung}
\author{Nico -telmich- Schottelius}
%\date{\today}
\date{5. September 2007}

\begin{document}

\frame{\titlepage}

%\section[Outline]{}
\frame{\tableofcontents}

% Inhalt:
% - Vorstellung
%  telmich
%  !eof
% - Motivation
%  * !eof befindet ist zur Zeit im IRCNet
%  * Etwas unangenehnem
%  * abhoerbar an vielen Stellen
% - Alternativen
%  irc+ssl:
%     alle links?
%  - ssl-only kanaele
%  - auf den servern abhoerbar
% silc
% jabber
%  sympathische software
% tor
%
% EOF
%  in mehreren projektphasen
% PGP
% momentaner stand
% Webseite: EOF-1, www.eof.name/projects/eof-1/
% Libraries, client, client zum client
%
%

\section{Einleitung}
\subsection{Vorstellung}
\frame
{
  \frametitle{Erste Vorstellung}

  \begin{itemize}
  \item Erste Pr\"asentation
  \pause
  \item Hier im CCCZH
  \pause
  \item Kreative K\"opfe
  \pause
  \item Kritik bekommen
  \pause
  \item Anregen zum Mitdenken und -machen
  \end{itemize}
}
\frame
{
  \frametitle{!eof}

  \begin{itemize}
  \item Junge Hackergruppe
  \pause
  \item 2001 gegr\"undet
  \pause
  \item Fokus auf Freundlichkeit
  \pause
  \item Deutschland, Niederlande, Schweiz
  \pause
  \item IRCNet: !eof
  \pause
  \item https://www.eof.name/
  \end{itemize}
}

\frame
{
  \frametitle{Nico Schottelius}

  \begin{itemize}
  \item Entwickelt freie, quelloffene Software (FOSS)
  \pause
  \item Philosophie: Machen statt meckern
  \pause
  \item aka telmich
  \pause
  \item https://nico.schottelius.org/
  \end{itemize}
}

\frame
{
  \frametitle{EOF}

  \begin{itemize}
  \item Alternatives Chatsystem
  \pause
  \item Projekt von !eof
  \end{itemize}
}

\subsection{Motivation}
\frame
{
  \frametitle{Warum dieser Vortrag?}

  \begin{itemize}
  \item Idee vorstellen
  \pause
  \item Kritik bekommen
  \pause
  \item Werben von
  \pause
  \begin{itemize}
  \item Entwicklern
  \pause
  \item Testern
  \pause
  \item (noch) nicht Benutzern
  \end{itemize}
  \end{itemize}
}
\frame
{
  \frametitle{Warum das Rad neu erfinden?}

  \begin{itemize}
  \item IRC ist grunds\"atzlich unverschl\"usselt
  \pause
  \item Suchen eine generell nicht abh\"orbare Alternative
  \pause
  \item Existierende Alternativen erf\"ullen nicht alle Anforderungen
  \end{itemize}
}
\frame
{
  \frametitle{Anforderungen}

  \begin{itemize}
  \item Dezentralit\"at
  \pause
  \begin{itemize}
  \item Ohne zentralen Server
  \pause
  \end{itemize}
  \item Nicht abh\"orbar
  \pause
  \begin{itemize}
  \item \textbf{Wer} mit \textbf{wem} \textbf{wann} \textbf{was} redet
  \pause
  \end{itemize}
  \item Verschl\"usselung und Signatur
  \end{itemize}
}

\subsection{\"Uber diesen Vortrag}
\frame
{
  \frametitle{Versionen dieses Vortrages}

  \begin{itemize}
  \item Nur Text (\LaTeX{} oder PDF)
  \pause
  \item Nur Ton (Aufnahme in Ogg)
  \pause
  \item Text und Ton (Theora-Video)
  \end{itemize}
}

\frame
{
  \frametitle{Kopieren dieses Vortrages}

  \begin{itemize}
  \item Ist (hiermit) explizit gestattet
  \pause
  \item Creative Commons Attribution-Share Alike 2.0 Germany License
  \pause
  \item http://creativecommons.org/licenses/by-sa/2.0/de/
  \end{itemize}
}

\frame
{
  \frametitle{Quellen dieses Vortrages}

  \begin{itemize}
  \item Unter folgenden Adressen zu finden:
  \pause
  \begin{itemize}
  \item https://nico.schottelius.org/projects/eof-1/
  \pause
  \item https://www.eof.name/projects/eof-1/
  \end{itemize}
  \end{itemize}
}

\section{Alternativen}
\subsection{IRC+SSL}
\frame
{
  \frametitle{\"Ubersicht IRC+SSL}

  \begin{itemize}
  \item Standard IRC mit SSL
  \pause
  \item Verschl\"usselung
  \begin{itemize}
  \item Client zum Server
  \pause
  \item Server zum Server
  \pause
  \end{itemize}
  \item "`Nur SSL"'-R\"aume
  \end{itemize}
}

\frame
{
  \frametitle{Vorteile}

  \begin{itemize}
  \item Erprobtes Protokoll
  \pause
  \item Stabile Software vorhanden
  \pause
  \item Verschl\"usselung ist einfach hinzuf\"ugbar
  \end{itemize}
}

\frame
{
  \frametitle{Nachteile}

  \begin{itemize}
  \item Innerhalb der Server ungesichert
  \pause
  \begin{itemize}
  \item Kompromitierung durch \"Ubernahme eines Servers m\"oglich
  \pause
  \end{itemize}
  \item Keine Ende-zu-Ende-Verschl\"usselung
  \pause
  \item Teile der Verbindung k\"onnen unverschl\"usselt sein
  \pause
  \item Statistische Analyse m\"oglich
  \pause
  \begin{itemize}
  \item Wer sendet wann, wer empf\"angt wann?
  \end{itemize}
  \end{itemize}
}

\subsection{SILC}
\frame
{
  \frametitle{\"Ubersicht SILC}

  \begin{itemize}
  \item \textbf{S}ecure \textbf{I}nternet \textbf{L}ive \textbf{C}hat
  \pause
  \item Geplanter Ersatz f\"ur IRC
  \pause
  \item Verschl\"usselung
  \pause
  \begin{itemize}
  \item Server zum Server
  \pause
  \item Client zum Server
  \pause
  \item Client zum Client
  \pause
  \end{itemize}
  \item http://www.silcnet.org/
  \end{itemize}
}

\frame
{
  \frametitle{Vorteile}

  \begin{itemize}
  \item Verschl\"usselung von Anfang eingeplant
  \pause
  \item Integration in einen bekannten Chat-Client (irssi)
  \end{itemize}
}

\frame
{
  \frametitle{Nachteile}

  \begin{itemize}
  \item Instabile Software
  \pause
  \begin{itemize}
  \item Client st\"urzt ab
  \pause
  \item Server verliert Kanal-Parameter (z.B. +r)
  \pause
  \item Server verliert Kan\"ale
  \pause
  \end{itemize}
  \item Eigenes (unerforschtes) Protokoll
  \pause
  \item Statistische Analyse m\"oglich
  \end{itemize}
}

\subsection{Jabber}
\frame
{
  \frametitle{\"Ubersicht Jabber}

  \begin{itemize}
  \item Instant messaging (IM) Protokoll
  \pause
  \item Transport via XML
  \end{itemize}
}

\frame
{
  \frametitle{Vorteile}

  \begin{itemize}
  \item Viele Clients verf\"ugbar
  \end{itemize}
}

\frame
{
  \frametitle{Nachteile}

  \begin{itemize}
  \item Kan\"ale sind eine Erweiterung
  \pause
  \item Verschl\"usselung optional
  \pause
  \item Statistische Analyse m\"oglich
  \end{itemize}
}

\subsection{Tor}
\frame
{
  \frametitle{\"Ubersicht Tor}

  \begin{itemize}
  \item Anonymisierungsdienst
  \pause
  \item Kein Chatprotokoll
  \pause
  \item Zwiebelprinzip
  \end{itemize}
}

\frame
{
  \frametitle{Vorteile}

  \begin{itemize}
  \item Stabile Software vorhanden
  \end{itemize}
}

\frame
{
  \frametitle{Nachteile}

  \begin{itemize}
  \item Client sendet nur wenn er wirklich sendet
  \pause
  \begin{itemize}
  \item Statistische Analyse m\"oglich
  \pause
  \end{itemize}
  \item Empf\"anger sendet Paket nicht weiter
  \pause
  \begin{itemize}
  \item Endpunkte sind erkennbar
  \end{itemize}
  \end{itemize}
}


\section{EOF}
\frame
{
  \frametitle{EOF}

  \begin{itemize}
  \item Projektaufbau
  \pause
  \item Grundideen
  \pause
  \item Beispiele
  \pause
  \item Stand
  \end{itemize}
}

\subsection{Projektaufbau}
\frame
{
  \frametitle{Evolution}

  \begin{itemize}
  \item Kleine Teile programmieren
  \pause
  \item Vorschl\"age f\"ur Protokolle entwerfen
  \pause
  \item Ausprobieren und aus Fehlern lernen
  \end{itemize}
}

\frame
{
  \frametitle{EOF-1}

  \begin{itemize}
  \item Erste Phase
  \pause
  \item Ideen dokumentieren
  \pause
  \item Ideen diskutieren
  \pause
  \item Viele Fehler machen
  \pause
  \item Juli 2007 bis Dezember 2007
  \end{itemize}
}

\frame
{
  \frametitle{EOF-1 Vorgehen}

  \begin{itemize}
  \item Spezifikationsentw\"urfe schreiben f\"ur
  \pause
  \begin{itemize}
  \item Protokolle
  \pause
  \item Software
  \pause
  \end{itemize}
  \item Freie Wahl der Programmiersprache
  \pause
  \begin{itemize}
  \item Kommunikation \"uber klar definierte Schnittstellen
  \end{itemize}
  \pause
  \item Modularer Aufbau
  \end{itemize}
}

\frame
{
  \frametitle{EOF-2}

  \begin{itemize}
  \item Fehler aus EOF-1 korrigieren
  \pause
  \item Dokumentation erstellen
  \pause
  \item Plan f\"ur finale Version erstellen
  \pause
  \item Januar 2008 bis Juni 2008
  \end{itemize}
}

\frame
{
  \frametitle{EOF-3}

  \begin{itemize}
  \item Spezifikation von EOF-2 implementieren
  \pause
  \item Unterst\"utzung mehrerer Betriebsysteme
  \pause
  \item Produktiv nutzbar
  \pause
  \item Juli 2008 bis Dezember 2008
  \end{itemize}
}

\subsection{Grundideen}
\frame
{
  \frametitle{Modularit\"at}

  \begin{itemize}
  \item Viele kleine Teile
  \pause
  \item Alles ist austauschbar
  \pause
  \item Klar definierte Schnittstellen
  \pause
  \item F\"ur anderes wiederverwendbar
  \end{itemize}
}


\frame
{
  \frametitle{Das Zwiebelprinzip}

  \begin{itemize}
  \item Mehrfach verschl\"usselt
  \pause
  \item Nie direkt an den Empf\"anger senden
  \pause
  \item Auch der Empf\"anger sendet es weiter!
  \end{itemize}
}

\frame
{
  \frametitle{Indirekte Kommunikation}

  \begin{itemize}
  \item \"Uber mehrere Stationen
  \pause
  \item Jede Station ent- und verschl\"usselt
  \pause
  \item Keine Station sendet direkt an den End-Empf\"anger
  \end{itemize}
}


\frame
{
  \frametitle{Alles verschl\"usselt}

  \begin{itemize}
  \item Kein unverschl\"usseltes Paket
  \pause
  \item Initialer Schl\"usseltausch \"uber sichere Kan\"ale
  \pause
  \item Schl\"usseltausch sp\"ater \"uber etablierten Kanal
  \end{itemize}
}


\frame
{
  \frametitle{Nachrichten signiert}

  \begin{itemize}
  \item Sicherstellen, dass ich mit dem richtigen kommuniziere
  \pause
  \item Signatur ist nur sichbar f\"ur den Empf\"anger
  \end{itemize}
}


\frame
{
  \frametitle{Abstraktion vom Transportprotokoll}

  \begin{itemize}
  \item \"Ubertragung ist \textbf{nicht} Bestandteil von EOF
  \pause
  \item Das Paketformat wird unabh\"angig vom Transportprotkoll definiert
  \pause
  \item Beliebige Protokolle m\"oglich
  \pause
  \item Einfaches Tunneln von Firewalls
  \pause
  \begin{itemize}
  \item z.B. direkte "`HTTP-Verbindung"' oder via Webforen
  \end{itemize}
  \end{itemize}
}


\frame
{
  \frametitle{Rauschen}

  \begin{itemize}
  \item Jeder Teilnehmer sendet "`immer"' (fixer Intervall)
  \pause
  \begin{itemize}
  \item Fest definierter Sendeintervall
  \pause
  \item Wenn nichts zu senden ist werden Zufallsdaten gesendet
  \pause
  \item Zufallsdaten werden genauso wie richtige Pakete behandelt
  \pause
  \end{itemize}
  \item Von ausen nicht zu erkennen, wann er wirklich sendet
  \end{itemize}
}


% beispiele
% - chatten mit wem direkt
% - key exchange: nur ueber bekannte peers
% - transport protokolle
% - Vollstaendige Verschluesselung
\subsection{Beispiele}
\frame
{
  \frametitle{Beispiele}

  \begin{itemize}
  \item F\"ur
  \pause
  \begin{itemize}
  \item Ausgew\"ahlte Teile von EOF
  \pause
  \item \"Ubliche Handlungen beim Chatten
  \pause
  \end{itemize}
  \item Um abstrakte Informationen konkretisieren
  \end{itemize}
}

\frame
{
  \frametitle{Wie man sich findet}

  \begin{itemize}
  \item \"Uber Marktschreier
  \pause
  \item Marktschreier verwalten Metaninformationen
  \pause
  \begin{itemize}
  \item Liste von bekannten Kan\"alen
  \pause
  \item Liste von bekannten Partner
  \pause
  \end{itemize}
  \item Jeder Client kann Marktschreier sein
  \end{itemize}
}

\frame
{
  \frametitle{\"Uber die Marktschreier}

  \begin{itemize}
  \item Besitzen und Verteilen \textbf{nur} \"offentliche Informationen
  \pause
  \item Verwalten Liste von \"offentlichen Kan\"alen
  \pause
  \item Vermitteln Partner zum Indirekten Senden
  \pause
  \item Sind ganz normale EOF-Clients
  \pause
  \begin{itemize}
  \item Erlauben zus\"atzlich Abfragen und Speichern von Metainformationen
  \end{itemize}
  \end{itemize}
}

\frame
{
  \frametitle{Vollstaendige Verschl\"usselung}

  \begin{itemize}
  \item \"Offentlicher Schl\"ussel wird vorher per
  \pause
  \begin{itemize}
  \item PGP verschl\"usselter E-Mail
  \pause
  \item Versiegelten Brief
  \pause
  \item Telefon
  \end{itemize}
  \item ausgetauscht
  \end{itemize}
}

\frame
{
  \frametitle{Beitreten eines Kanals}

  \begin{itemize}
  \item Verbinden zum Marktschreier
  \pause
  \begin{itemize}
  \item "`Was f\"ur Kan\"ale kennst Du?"'
  \pause
  \item Liste: "`blackhats"' ... "`antim\$"' ... "`EOF-1"' ...
  \pause
  \item "`Teile bitte den Leuten von Kanal EOF-1 mit, dass ich beitreten m\"ochte."'
  \pause
  \end{itemize}
  \item Marktschreier verbindet sich zu einem Partner, der die anderen kennt
  \pause
  \item Der wiederrum schickt Pakete an alle anderen (indirekt) weiter
  \pause
  \item Die Kanalinsassen melden sich dann bei mir
  \end{itemize}
}

\frame
{
  \frametitle{Woher wissen die, wie sie mich erreichen?}

  \begin{itemize}
  \item Zum Anfang \"ubermittelt man dem Marktschreier, wo man erreichbar ist
  \pause
  \item Diese Information leitet er weiter
  \end{itemize}
}

\frame
{
  \frametitle{Transport-Protokolle}

  \begin{itemize}
  \item Etwas, mit dem man senden und/oder empfangen kann
  \pause
  \item Beliebige Protokolle
  \pause
  \begin{itemize}
  \item Empfangen via SMTP HELO
  \pause
  \item Senden via Webdav
  \pause
  \item Empfangen via ICQ/Jabber/Skype/...
  \end{itemize}
  \end{itemize}
}

\frame
{
  \frametitle{Indirekte Kommunikation}

  \begin{itemize}
  \item Wenn ich mit Andre reden will, verschl\"ussele ich die Nachricht f\"ur ihn
  \pause
  \item Und f\"uge eine Liste von Empf\"angern zur Nachricht hinzu
  \pause
  \item Jeder der Empf\"anger kann (nur) die n\"achste Empfangsadresse sehen
  \pause
  \item Andre
  \pause
  \begin{itemize}
  \item entschl\"usselt die Nachricht, 
  \pause
  \item schickt sie weiter, damit niemand weiss, das sie f\"ur ihn war
  \end{itemize}
  \end{itemize}
}

\subsection{Stand}
\frame
{
  \frametitle{Stand}

  \begin{itemize}
  \item Was wurde bereits erledigt
  \pause
  \begin{itemize}
  \item Spezifikationen
  \pause
  \item Programme
  \pause
  \item Dokumentation
  \pause
  \end{itemize}
  \item Was ist zu tun?
  \begin{itemize}
  \pause
  \item Meilensteine
  \end{itemize}
  \end{itemize}
}

\frame
{
  \frametitle{Spezifikationen}

  \begin{itemize}
  \item Einige Entw\"urfe auf https://www.eof.name/projects/eof-1/eof-1/ zu finden
  \pause
  \item W\"ochentlich neue
  \pause
  \item Teilweise schon \"uberarbeitet
  \end{itemize}
}

\frame
{
  \frametitle{Implementation von EOF-1}

  \begin{itemize}
  \item Hauptprogramm ceof zu 20\% fertig
  \pause
  \item Erste Erfahrungen mit GPGME 
  \pause
  \item Bibliotheken zu 80\% fertig
  \pause
  \begin{itemize}
  \item Quelltexts\"auberung steht noch an
  \pause
  \end{itemize}
  \item Erste GUI fertiggestellt
  \end{itemize}
}

\frame
{
  \frametitle{Von Dir}

  \begin{itemize}
  \item Verwirrt?
  \pause
  \item Ist etwas unklar?
  \pause
  \item Interessiert?
  \end{itemize}
}

\frame
{
  \frametitle{Diese Vorstellung}

  \begin{itemize}
  \item ...endet hier
  \pause
  \item Vielen Dank f\"ur die Aufmerksamkeit
  \pause
  \item Projektseite: https://www.eof.name/eof-1/
  \pause
  \item Kontakt via "`telmich (bei) u.eof.name"'
  \end{itemize}
}


\end{document}
