\documentclass[12pt,a4paper]{article}
% FIXME: UTF8
\usepackage[latin1]{inputenc}    % Ascii-Format dieses Dokuments
\usepackage{longtable}           % lange Tabellen
%\usepackage[dvips]{epsfig}

\begin{document}
\title{gui2ceof}
\date{2007-09-16 v0.2-train}
\author{telmich}

\maketitle
\tableofcontents

\section{Introduction}
This document specifies the commands send from GUI to and from
ceof\footnote{the central EOF-1 application} to the GUIs.

\subsection{Changelog}
\subsubsection{draft-1 to v0.2-mrmcd110b}
\begin{itemize}
\item Changed socket location
\item Clearified what todo without \textit{CEOF\_DIR} and \textit{HOME} set
\item Added exit command.
\item Changed number of bytes for marktschreier from 512 Bytes to 128 Bytes
\item Changed number of bytes for channel name from 512 Bytes to 128 Bytes
\end{itemize}

TODO: how client gets peer ids? necessary?
retrieve list of channels,
retrieve list of peers / peer-ids.

mapping from peer-id to real id.
channel list: known channels on marktschreier,
joined channels;
members of channel;
add notification of join request;

\section{Connection}
The client connects to a socket named \textit{\$HOME/.ceof/clients/socket}.
If the environment variable "`\textit{CEOF\_DIR}`" is set,
\textit{\$HOME/.ceof} should be replaced with that content.
If the environment variable \textit{CEOF\_CLIENT\_SOCKET} is set,
"`\textit{clients/socket}"` should be replaced by its content.
If the environment variable \textit{HOME} and \textit{CEOF\_DIR}
is empty, you should fallback to the directory .ceof in the currenty directory.

% -----------------------------------------------------------------------------
\section{Commands}
All commands are send as 4-Byte ASCII digits (for instance "`0012"').
All answers and all numbers are ASCII-numbers. We
\textbf{never} transmit binary numbers.
\textbf{Client commands} always begin with \textbf{0} ("`0042"' for instance),
\textbf{answers} or \textbf{notifications} from
ceof begin with \textbf{1} ("`1023"' for instance).
After each command follows individual data. The second byte indicates the type of message:
\begin{itemize}
\item \textbf{00}: client meta command (something that does not affect the user)
\item \textbf{01}: messages
\item \textbf{11}: sucess answers from ceof
\item \textbf{12}: error answers from ceof
\item \textbf{13}: messages / notifications initiated by ceof
\end{itemize}

% -----------------------------------------------------------------------------
\subsection{0000: Register client}
After the "`0000"' the client directly appends two ASCII digits containing the
version of the client to ceof protocol it speaks. This specification
uses version "`02"'.
Answers from ceof:
\begin{itemize}
\item \textbf{1100}: sucess, you are connected
\item \textbf{1200}: version not supported
\end{itemize}
% OK
% -----------------------------------------------------------------------------

% -----------------------------------------------------------------------------
\subsection{0001: connect to markschreier}
After the "`0001"' follow 128 Bytes describing how to connect
to the marktschreier ("`tcp://62.65.138.66:42'" for instance).
If the URI is shorter than 128 bytes, the remain should be filled with 0 bytes
(also known as \verb='\0'=).

Answers from ceof:
\begin{itemize}
\item \textbf{1100}: success
\begin{itemize}
\item After the 1100 follow four ASCII digits containing the
internal marktschreier id
\end{itemize}
\item \textbf{1201}: high level protocol (like "`tcp://"') not supported
\item \textbf{1202}: connection could not be established
\end{itemize}
% OK
% -----------------------------------------------------------------------------
\subsection{0002: get list of channels}
After the "`0002"' follow 4 ASCII Bytes containing the internal
markschreier id.

Answers from ceof:

\begin{itemize}
\item \textbf{1101}: got list; following four ASCII-digits containing
number of channels ("`num\_chan"'); after that  num\_chan 128 Bytes packets
follow containing the channel name, padded with 0-bytes
\item \textbf{1202}: connection could not be established
\end{itemize}
% OK
% -----------------------------------------------------------------------------

\subsection{0003: ask to join a channel}
After the "0003" follow 128 Bytes describing the channel name.
Answers from ceof:
\begin{itemize}
\item \textbf{1100}: success (means: markschreier asked knows peers to
connect to us); 4 Bytes follow, giving the internal channel id
\item \textbf{1202}: connection could not be established
\item \textbf{1203}: access denied by markschreier: you are not allowed to join
\end{itemize}
% -----------------------------------------------------------------------------
\subsection{0004: send message to channel}
Aftere the "0004" follow four bytes for the ceof internal channel id and after
that 512 Bytes for the message.
Answers from ceof:
\begin{itemize}
\item \textbf{1100}: success
\item \textbf{1204}: unknown channel
\end{itemize}
% -----------------------------------------------------------------------------
\subsection{0005: send message to peer}
Aftere the "`0005"' follow four bytes for the ceof internal peer
id and after that 512 Bytes for the message.
Answers from ceof:
\begin{itemize}
\item \textbf{1100}: success
\item \textbf{1205}: unknown peer
\end{itemize}
% -----------------------------------------------------------------------------
\subsection{0006: retrieve list of known peers}
Answers from ceof:
\begin{itemize}
\item \textbf{1102}: list follows; following four ASCII-digits containing number
of peers ("`num\_peer"'); after that  num\_peer
XXX Bytes packets follow containing the peer name, real peer id, padded with 0-byte
\end{itemize}
% -----------------------------------------------------------------------------
\subsection{0099: request for exit}
Tells ceof to exit and to notify all clients to exit. It will not reply anything to you, but issue an exit notify to all clients, including the requesting one.
% -----------------------------------------------------------------------------
\subsection{1300: recieved message}
After the "1300" follows:
\begin{itemize}
\item 4 Byte channel id; if the id is "`9999"' it is a private message, directly send to you
\item 4 Byte peer id (the person who send the message)
\item 512 Byte message
\end{itemize}
% -----------------------------------------------------------------------------
\subsection{1399: exit notify}
ceof is being shutdown.
Shutdown yourself, too.
After that message ceof will exit and you should do the same.
No answer possible, ceof already decided to vanish.
% -----------------------------------------------------------------------------
\section{The way it works}
This section explains how the commands relate together and which commands to use
in which order.

% -----------------------------------------------------------------------------
\subsection{GUI initiated commands}
% -----------------------------------------------------------------------------
\subsubsection{GUI startup}
What todo, when the GUI starts.
Connect to ceof. Find out about
\begin{itemize}
\item joined channels,
\item connected marktscheier
\item and open queries.
\end{itemize}
It thus issues the following commands:
register, list joined channels, list open queries, list marktschreier.
% -----------------------------------------------------------------------------
\subsubsection{Creating a channel}
When you want to create a channel, you simply have to give it a name.
Ceof will use that name and sign it with your pgp key. The result is the
global unique channel identifier. For testing, you can build your channels
easily on the commandline:
\begin{verbatim}
% echo -n '!eof' > CHANNELNAME
% cat CHANNELNAME 124byteszero > CHANNELNAME.padded
% gpg -s CHANNELNAME 

You need a passphrase to unlock the secret key for
user: "Nico Schottelius (telmich) <nico-public@schottelius.org>"
1024-bit DSA key, ID 9885188C, created 2006-09-27

% ls -l CHANNELNAME*
-rw------- 1 nico nico   4 2007-09-17 21:32 CHANNELNAME
-rw------- 1 nico nico 128 2007-09-17 21:32 CHANNELNAME.padded
-rw------- 1 nico nico 113 2007-09-17 21:32 CHANNELNAME.padded.gpg

% gpg -d CHANNELNAME.padded.gpg 2>/dev/null
!eof
\end{verbatim}
% -----------------------------------------------------------------------------
\subsubsection{Finding a channel}
replace internal id with signature!

The GUI asks ceof, which channels are known. After that the GUI
can use the internal ID (which is "`unique"', as in: it is a 32 bit integer
that is increased as long as ceof is running) to join it.
% -----------------------------------------------------------------------------
\subsubsection{Joining a channel}
The channel must be known through 
% -----------------------------------------------------------------------------
\subsubsection{Leaving a channel}
Sends a message to all known participants that we leave the channel now.
% -----------------------------------------------------------------------------
\subsubsection{Invite to channel}
If you create a new channel, you may want to invite people to it.
% -----------------------------------------------------------------------------
\subsubsection{Listing friends}
% -----------------------------------------------------------------------------
\subsection{ceof initiated commands / messages}
% -----------------------------------------------------------------------------
\subsubsection{Recieved a join request}
Somebody wants to join in a channel, in which you are a member.
% -----------------------------------------------------------------------------
\subsubsection{Recieved a join notification request}
Somebody wants us to tell all members of the channel that she wants to join us.
% -----------------------------------------------------------------------------
\subsubsection{Recieved a channel message}
Must contain which channel, who send it and what is in the message.
% -----------------------------------------------------------------------------
\subsubsection{Recieved a private message}
This messages was send only to us, not to a channel.
% -----------------------------------------------------------------------------
\end{document}
