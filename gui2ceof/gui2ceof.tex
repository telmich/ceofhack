\documentclass[12pt,a4paper]{article}
% FIXME: UTF8
\usepackage[latin1]{inputenc}    % Ascii-Format dieses Dokuments
\usepackage{german}
\usepackage{longtable}           % lange Tabellen
%\usepackage[dvips]{epsfig}

\begin{document}
\title{gui2ceof}
\date{2007-09-15 v0.2-mrmcd110b}
\author{telmich}

\maketitle

\section{Introduction}
Missing.- Commands from GUI to and from
ceof\footnote{the central EOF-1 application}.

How the clients can communicate with ceof, the central application of EOF-1.

\subsection{Changelog}
Changes to draft-1: Change socket location, clearified what todo without CEOF_DIR and HOME. Added exit command. TODO: how client gets peer ids? necessary? retrieve list of channels, retrieve list of peers / peer-ids. mapping from peer-id to real id. channel list: known channels on marktschreier, joined channels; members of channel; add notification of join request;

\section{Connection}
The client connects to a socket named \verb=$HOME/.ceof/clients/socket=.
If the environment variable "`\textit{CEOF_DIR}`" is set,
\verb=$HOME/.ceof= should be replaced with that content.
If the environment variable \verb=$CEOF_CLIENT_SOCKET= is set,
"`\textit{clients/socket}"` should be replaced by its content.
If \verb=$HOME= and \verb=$CEOF_DIR= is empty, you should fallback
 to the directory .ceof in the currenty directory.

\section{Commands}
All commands are send as 4-Byte ASCII digits (for instance "`0012"').
All answers and all numbers are ASCII-numbers. We never transmit binary numbers.
\textbf{Client commands} always begin with "`0"' ("`0042"' for instance),
\textbf{answers} or notifications from ceof begin with "`1"' ("`1023"' for instance).
 After each command follows individual data. The second byte indicates the type of message:
\begin{itemize}
\item 00: client meta command (something that does not affect the user)
\item 01: messages
\item 11: sucess answers from ceof
\item 12: error answers from ceof
\item 13: messages / notifications initiated by ceof
\end{itemize}

\subsection{0000: Register client}

After the "0000" the client directly appensd two ASCII digits containing the version of the client to ceof protocol it speaks ("01" in this case).

Answers from ceof:

    * 1100: sucess, you are connected
    * 1200: version not supported


\subsection{0001: connect to markschreier}

After the "0001" follow 512 Bytes describing how to connect to the marktschreier ("tcp://62.65.138.66:42" for instance). If the URI is shorter than 512 bytes, the remain should be filled with 0 bytes (also known as '\0').

Answers from ceof:

    * 1100: success
    * 1201: high level protocol (like "tcp://") not supported
    * 1202: connection could not be established


\subsection{0002: get list of channels}

Answers from ceof:

    * 1101: got list; following four ASCII-digits containing number of channels ("num_chan"); after that  num_chan 512 Bytes packets follow containing the channel name, padded with 0-bytes
    * 1202: connection could not be established


\subsection{0003: ask to join a channel}

After the "0003" follow 512 Bytes describing the channel name.

Answers from ceof:

    * 1100: success (means: markschreier asked knows peers to connect to us); 4 Bytes follow, giving the internal channel id
    * 1202: connection could not be established
    * 1203: access denied by markschreier: you are not allowed to join


\subsection{0004: send message to channel}

Aftere the "0004" follow four bytes for the ceof internal channel id and after that 512 Bytes for the message.

Answers from ceof:

    * 1100: success
    * 1204: unknown channel


\subsection{0005: send message to peer}

Aftere the "0004" follow four bytes for the ceof internal peer id and after that 512 Bytes for the message.

Answers from ceof:

    * 1100: success
    * 1205: unknown peer


\subsection{0006: retrieve list of known peers}

Answers from ceof:

    * 1102: list follows; following four ASCII-digits containing number of peers ("num_peer"); after that  num_peer  XXX Bytes packets follow containing the peer name, real peer id, padded with 0-byte

\subsection{0099: request for exit}

Tells ceof to exit and to notify all clients to exit. It will not reply anything to you, but issue an exit notify to all clients, including the requesting one.


\subsection{1300: recieved message}

After the "1300" follows:

    * 4 Byte channel id; if the id is "9999" it is a private message, directly send to you
    * 4 Byte peer id (the person who send the message)
    * 512 Byte message


\subsection{1399: exit notify}

ceof is being shutdown. Shutdown yourself, too. After that message ceof will exit and you should do the same. No answer possible, ceof already decided to vanish.
\end{document}
