\documentclass[12pt,a4paper]{article}
% FIXME: UTF8
\usepackage[latin1]{inputenc}    % Ascii-Format dieses Dokuments
\usepackage{longtable}           % lange Tabellen
%\usepackage[dvips]{epsfig}

\begin{document}
\title{politiker2ceof}
\date{2007-11-11 v0.1-train}
\author{telmich}

\maketitle
\tableofcontents

\section{Introduction}
This document specifies the commands send from politiker to and from
ceof\footnote{the central EOF-1 application} to the politiker.

\subsection{Changelog}
\subsubsection{none}
\begin{itemize}
\item -
\end{itemize}
% -----------------------------------------------------------------------------
\section{Connection}
The politiker is started by ceof at startup and communicates through stdin
and stdout.
% -----------------------------------------------------------------------------
\section{Commands}
All commands are send as uint32\_t types.
\textbf{Politiker commands} always begin with
\textbf{3} ("`3042"' for instance),
\textbf{answers} or \textbf{notifications} from
ceof begin with \textbf{1} ("`1023"' for instance).
After each command follows individual data. The second byte indicates the type of message:
\begin{itemize}
\item \textbf{30}: politiker messages
\begin{itemize}
\item \textbf{300}: (De-)Initialisation
\item \textbf{301}: Peer related messages
\item \textbf{302}: Message related messages
\end{itemize}
\item \textbf{10}: ceof messages
\begin{itemize}
\item \textbf{100}: (De-)Initialisation
\item \textbf{101}: Peer related messages
\item \textbf{102}: Message related messages
\item \textbf{103}: Message related answers
\end{itemize}
\end{itemize}

% -----------------------------------------------------------------------------
\subsection{30: Politiker messages}
% -----------------------------------------------------------------------------
\subsection{3000: Register politker}
After the "`3000"' the politker directly appends an
\textit{uint32\_t} containing the version of the politker to ceof protocol
it speaks. This specification uses version number "`0"'.
Answers from ceof:
\begin{itemize}
\item \textbf{1100}: sucess, you are connected
\item \textbf{1200}: version not supported
\end{itemize}
% -----------------------------------------------------------------------------
\subsection{3001: Deregister politiker}
The politker has some problem and has to exit. Ceof will restart a new
instance of it.
Answers from ceof:
\begin{itemize}
\item none
\end{itemize}
% -----------------------------------------------------------------------------
\subsection{3010: Retrieve random peer address and fingerprint}
The politker needs some peer information to be used as a hop.
Ceof forwards that request to \textbf{pmg}\footnote{peer manager} and returns
the answer to the politiker.
Answers from ceof:
\begin{itemize}
\item 1010: Data follows
\begin{itemize}
\item \textit{peer\_address}: 128 Bytes, 0 padded, 0 terminated
\item \textit{peer\_fingerprint}: 40 Bytes char array
\end{itemize}
\end{itemize}
% -----------------------------------------------------------------------------
\subsection{3011: Retrieve number of available peers}
The politker needs to know how much "`unique"' peers are available,
so it can match the required minimum.
Ceof forwards that request to \textbf{pmg} and returns the answer to the
politiker.
Answers from ceof:
\begin{itemize}
\item 1010: Data follows
\begin{itemize}
\item \textit{number\_of\_peers}: uint32\_t
\end{itemize}
\end{itemize}
% -----------------------------------------------------------------------------
\subsection{3020: Created an encrypted packet}
Passes the following information to ceof:
\begin{itemize}
\item The length of the packet (uint32\_t) (\textit{pck\_len})
\item The packet (\textit{pck})
\end{itemize}

After the \textbf{politiker} send \textit{pck\_len},
\textbf{ceof} must respond with either
\begin{itemize}
\item \textbf{1030}: pagket length is accepted
\item \textbf{1031}: pagket length is too long
\end{itemize}

If the response is \textit{1030}, \textbf{politiker} should send the
pcaket, otherwise this session is finished and \textbf{politiker}
should drop the packet.
% -----------------------------------------------------------------------------
\subsection{10: Ceof messages}
% -----------------------------------------------------------------------------
\subsection{1020: Create encrypted packet}
Passes the following information to the politiker:
\begin{itemize}
\item GPG-Fingerprint of the peer (40 Bytes char array) (\textit{fpr})
\item Adress of the peer (128 Bytes, 0 padded, 0 terminated) (\textit{address})
\item The length of the message (uint32\_t) (\textit{msg\_len})
\item The message (\textit{msg})
\end{itemize}

After \textbf{ceof} send \textit{msg\_len}, politiker must respond with
either
\begin{itemize}
\item \textbf{3020}: message length is accepted
\item \textbf{3021}: message length is too long
\end{itemize}

If the response is \textit{3020}, \textbf{ceof} should send the messsage,
otherwise this session is finished and the next thing the politiker expects
is a command.
% -----------------------------------------------------------------------------
\end{document}
